\documentclass{article}
\usepackage{graphicx} % Requerido para insertar imagénes. %


\begin{document}
%Para añadir imagenes: 
\begin{figure}[h] % Con corchetes, podemos añadr opciones de comando que nos permiten determinar la posición de la imagen, por ejemplo "[h](here)" muestra la imagen en el mismo lugar donde se inserta, o "[t](top), muestra la imagen en la parte superior de la página...
    \begin{center} % Entorno para centrar la imagen.
        \centering
        % El comando "width" indica el tamaño del ancho de la imagen.%
        \includegraphics[width=0.9\linewidth]{Donald_Trump_official_portrait.jpg}
       % El comando "caption" añade una leyenda en la parte inferior de la imagen.%
        \caption{Presidente de Estados Unidos}
        % El comando "label" es el nombre con el cual la imagen será recordada en los archivos (referencia cruzada).%
        \label{fig:Trump}
    \end{center}
\end{figure}
% Con el comando "ref" hacemos referencia a la imagen utilizando su label.%
El presidente de Estados unidos Donald Trump \ref{fig:Trump}....

\end{document}
