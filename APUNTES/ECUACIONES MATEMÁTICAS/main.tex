\documentclass{article}
\usepackage{graphicx} % Required for inserting images

\title{ECUACIONES MATEMÁTICAS}
\author{Giuseppe Fuentes Moreno}
\date{September 2024}

\begin{document}

\maketitle


La ecuación \ref{eq:1} es:
% Con el comando "\begin{equation}" se redaccta una ecuación, el "\label{eq:1}" nombra la ecuación.
\begin{equation}\label{eq:1}
    Y=X^2+F_c-\frac{a}{b}
\end{equation}
%La composición "$\alpha$", permite que no surgan problemas en la representaición de ecuaciones o carácteres en el interior de un texto.
Los valores de  $\alpha$ y $\beta$ son 0.1 y 0.2 respectivamente:
% A la hora de evaluar ecuaciones, lo más comodo es evitar la enumeración de la misma, para ello, escribimos la siguiente composición:
\[Y=5^2+4-\frac{0.1}{0.2}\]

% Material de apoyo para símbolos: "http://www.alciro.org/alciro/Matematicas-Web-LaTeX_14/Simbolos-Matematicos-LaTeX_103.htm" y formulas más comunes: "https://wiki.geogebra.org/es/Código_LaTeX_para_las_fórmulas_más_comunes". o "https://editor.codecogs.com"


\end{document}
