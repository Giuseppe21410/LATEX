% Para encontrar información de referencia, puedes utilizar el software "Mendeley Desktop" o "Google Academic".%

% Una vez obtenida las referencias en código "Bibtex", debemos crear un archivo ".bib" para almacenar nuestras referencias.%

% Existen difentes comándos "Bibtex", que puede rellenar a tu gusto. Todas ellas se encuentran en tus archivos. Un ejemplo de ello es el archivo "ifacconf.bib"

\documentclass{article}
\usepackage{graphicx} 
\usepackage{apacite}
\bibliographystyle{apacite}
\usepackage{natbib}

\title{BIBLIOGRAFÍAS}
\author{Giuseppe Fuentes Moreno}

\begin{document}
\section{Introducción}
%Para citar referencias utilizamos el código "\cite" para que el nombre de los autores aparezca en el texto o "\citep" para  dar una idea del trabajo de los autores consultados y no hemos mencionado en el texto:
El cemento fue descubierto en ... ,\citep{cemento}

Según \cite{cemento}...
%Para detallar la bibliografía utilizamos el código "bibliography":
\bibliography{referencias} % En donde detallamos el archivo referencias.bib.

\end{document}
