% Para copiar le código de una tabla de excel, debes seleccionar la tabla, ir a complementos y "convertir a Latex".%

\documentclass{article}
\usepackage{graphicx} 

\title{TABLAS DE EXCEL}
\author{Giuseppe Fuentes Moreno}
\date{September 2024}

\begin{document}

\maketitle

\section{Tabla de datos}

Los datos que se presentan en \ref{tab:addlabel}...

%Pegas el código y modificas el "label" de la tabla.

\begin{table}[htbp]
  \centering
  \caption{Add caption}
    \begin{tabular}{crr}
    \rowcolor[rgb]{ 0,  .69,  .941} \textbf{Datos} & \multicolumn{1}{c}{\textbf{(V)(cm3)/(mm3)}} & \multicolumn{1}{c}{\textbf{(m)(g)}} \\
    1     & 8,177083333 & 20,69921 \\
    2     & 8285,50 & 20,69921 \\
    \end{tabular}%
  \label{tab:addlabel}%
\end{table}%

% También existen plataformas en línea que crear directamente el código "Latex" de la tabla a medida que está siendo creada.
% https://www.tablesgenerator.com


\end{document}
