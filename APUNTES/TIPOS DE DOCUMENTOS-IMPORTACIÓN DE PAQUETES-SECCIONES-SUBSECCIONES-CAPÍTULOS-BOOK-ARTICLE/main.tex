% Existen diferentes tipos de documentos: "Article" (Para documentos cortos y artículos científicos), "report2" (Para documentos largos), "letter" (Para cartas), "book" (Para libros)...Además, podemos utilizar archivos ".cls" que definen la apariencia global de documento.%
\documentclass[11pt,onecolumn,a4paper,double]{article}
% Entre corchetes se muestran las opciones asociadas al documentos, por ejemplo:
% 1. 11 pt: Tamaño de letra de 11 puntos 
% 2. onecolumn: Una columna.
% 3. a4paper: Tamaño y formato a4.
% 4. double: Doble espaciado entre líneas.

% Al comienzo de cada artículo importamos los paquetes de uso:
\usepackage{graphicx} % Requerido para insertar imagénes.
\usepackage[utf8]{inputenc} %Requerido para tipografías de diferentes idiomas.
\usepackage{natbib}% Requerido para añadir un tipo de bibliografía.
\usepackage[spanish]{babel}% Requerido para añadir el diccionaria de español.%
%Para que el documento sea en español:
\selectlanguage{spanish}

% Iniciamos nuestro entorno de programación, todo el el documento se encuentra contenido en el mismo. Cabe resaltar que existen diferentes tipos de entorno, todos ellos se diferencian en su argumento de entrada "{...}".
\begin{document}
% El entorno del comando "\begin{frontmatter}", hace que las páginas estén enumeradas  y hace que los capítulos no estén enumerados, aunque el título de cada capítulo aparece en la tabla de contenidos.
%\begin{frontmatter}
    
% Al comienzo de cada documentos se inicia con una sección:
\section{Ejercicios:}
Me gusta el chocolate...
% Se pueden complentar con subsecciones:
\subsection{Ejercicio 1:}
Tienes que redactar...
\subsection{Ejercicio 2:}
% Se puede añadir una subsección más anteponiendo más "sub", al comando "\section", por ejemplo dos subsecciones se escribe como "\subsubsection{}"

% Además de secciones, para documentos de "\documentclass{book}", se utiliza "\chapter{}" para clasificar los contenidos en capítulos. Tienen más jerarquía que las secciones.

%\end{frontmatter}
\end{document}
